
\section{Data Collection}

To build the network we needed information on the votes of individual members of the Chamber. This kind of information is available starting from the XVII legislature (to 28 April 2023).


\subsection{Selected Data Sources}

Our main source of data is the official website of the Italian Chamber of Deputies. \footnote{\url{https://dati.camera.it/sparql}}



\subsubsection{Crawling Methodology and Assumptions}

To collect data we used a series of queries which, given a certain voting day, crawled the votes expressed by different members of the Chamber.
 To make it easier for our analysis, we aggregated the votes using different timescales, collecting a series of datasets for each of the following:

\begin{enumerate}
\item Votes for each legislature (XVII and XVIII).
\item Votes for each year (if a year falls in between two legislatures, we divide the dataset in two).
\item Votes for each month.
\end{enumerate}

\subsubsection{Preprocessing}

Our datasets showed that many Deputies were constantly absent from voting sessions. We decided to exclude them from our analysis since they would have mainly contributed to noise.\\

Specifically, we eliminated the Deputies which were absent in more than 70\% of the votings. 
