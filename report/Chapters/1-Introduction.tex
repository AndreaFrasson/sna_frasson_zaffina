\section{Introduction}

The Italian Chamber of Deputies is composed of 630 members\footnote{In 2020, a reform was introduced lowering this number to 400. This change entered into force in the XIX legislature, which we do not consider in our study.} who are elected for a five years term, this period is called legislature.\\
The political scene in the country is in constant development, and in recent years it was defined as tripolar\footnote{https://cise.luiss.it/cise/2022/09/29/landamento-dei-livelli-di-bipolarismo-e-bipartitismo/}: since 2013 the main coalitions are divided in center-right, center-left and 'Movimento 5 Stelle'.
However, during the years considered in this paper, new parties have emerged and different alliances have been formed, constantly changing the country's political equilibrium.\\

Previous work analyzed the behavior of political party members to identify how ideological communities are created and evolve over time \cite{AnIdeoComm} \cite{10.1371/journal.pone.0116046}, defining methodologies to analyze this kind of network in different political contexts .\\We decided to analyze data from the XVII and XVIII legislatures from April 2013 to October 2022. During these years, multiple governments have been in charge and different party alliances have been formed in the Parliament. We here summarize the governments which succeeded during these years\footnote{\url{https://www.senato.it/leg/ElencoMembriGoverno/Governi.html}}:

\begin{itemize}
\item XVII legislature
\begin{itemize}
    \item Letta-I (April 28 2013 - February 21 2014) 
    \item Renzi-I (February 22 2014 - December 11 2016)  
    \item Gentiloni Silveri-I (December 12 2016 - May 31 2018) 
\end{itemize}
\item XVIII legislature
\begin{itemize}
    \item Conte-I (June 1 2018 - September 4 2019) 
    \item Conte-II (September 5 2019 - February 12 2021) 
    \item Draghi-I (February 13 2021 - October 21 2022) 
\end{itemize}
\end{itemize}

This paper's first section explains how we collected data using the voting sessions and converted them into weighted networks. Also, some descriptive analyses are presented to describe the structure of these networks. Then, we compare the results of different community detection algorithms, emphasizing the differences in relation to the 'big coalitions'. 
Then we study the monthly evolution of the network, using a stream graph approach. And finally, we look for patterns in the network features that could relate to a government crisis.\\
